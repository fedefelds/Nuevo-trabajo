\documentclass[]{article}
%\documentclass[journal,10pt,draftclsnofoot,onecolumn]{IEEEtran}
%\usepackage{graphics,multirow,amsmath,amssymb,textcomp,subfigure,multirow,xspace,arydshln,cite}

\usepackage[]{graphicx}   % para manejar graficos

\usepackage[space]{grffile} % para manejar graficos

\usepackage{caption}

\usepackage{enumerate}    % para hacer listas numeradas

\usepackage{amsmath}        % no se..

\usepackage{amsfonts}     % no se..

\usepackage{authblk}    % para definir las afiliaciones de cada autor

\usepackage{layout}     % no se..

\usepackage[sorting=none,backend=bibtex]{biblatex}   % para manejar la bibliografia / referencias

\usepackage{lipsum}     % para generar texto random

\usepackage{multicol}   % para usar dos columnas

\usepackage{palatino}   % para que la fuente sea palatino

\usepackage[utf8]{inputenc} % para poder usar tildes

%\DeclareUnicodeCharacter{0301}{´} % para poder usar ``tildes externas''

\usepackage[spanish]{babel} % para escribir en español

\usepackage{csquotes}

\addto\captionsspanish{\def\tablename{Tabla}} % cambiar ``cuadro'' por ``Tabla''

\usepackage[sc,big,raggedright,bf]{titlesec} % para definir el formato del header de cada seccion.

\usepackage[font=small]{caption} % para que la fuente de un epigrafe no tenga el mismo tamaño que el cuerpo del texto

\usepackage{geometry}
 \geometry{
 a4paper,
 textwidth={17cm},
 textheight={23cm},
 left={2cm},
 top={2.5cm},
 }

\setlength{\columnsep}{1cm} % para que la separacion entre columnas sea de 1 cm

\graphicspath {{imagenes/}}

\defbibheading{bibliography}{\section{\refname}} % para que bibtex no imponga su header cuando uso \printbibliography, y que se use el de babel

\addbibresource{bibliografia.bib} % para importar el archivo .bib

\title{\textbf{\LARGE{\textsf{MEDICIÓN DE ESPECIFICACIONES DE PARLANTES}}}}
 % defino el titulo del Paper

\date{} % lo pongo vacio para que no aparezca abajo del abstract

\newcommand{\figura}[3]{
% \vspace{0.2 cm}
\begin{figurehere}
\centering
\includegraphics[width=\linewidth]{#1}
\captionof{figure}{#2}
\label{#3}
\end{figurehere}
% \vspace{0.2 cm}
}

\newcommand{\tabla}[4]{
\vspace{0.3 cm}
\begin{tablehere}
\begin{center}
\begin{tabular}{#1}
#2
\end{tabular}
\caption{#3}
\label{#4}
\end{center}
\end{tablehere}
\vspace{0.3 cm}
}

\usepackage{fancyhdr}
\usepackage{hyperref}
%%%%%%%%%%%%%%%%%%%%%%%%%%%%%%%%%%%%%%%%%%%%%%%%%%%%%%%%%%%%%%%%%%%%%%%%%%%%%%%%%
% http://www-h.eng.cam.ac.uk/help/tpl/textprocessing/multicol_hint.html
\makeatletter           % esto lo uso para poder definir figuras
\newenvironment{tablehere}    % esto lo uso para poder definir figuras
  {\def\@captype{table}}    % esto lo uso para poder definir figuras

  {}              % esto lo uso para poder definir figuras
                  % esto lo uso para poder definir figuras
\newenvironment{figurehere}   % esto lo uso para poder definir figuras
  {\def\@captype{figure}}   % esto lo uso para poder definir figuras
  {\par\medskip}
  {}              % esto lo uso para poder definir figuras
\makeatother          % esto lo uso para poder definir figuras
%%%%%%%%%%%%%%%%%%%%%%%%%%%%%%%%%%%%%%%%%%%%%%%%%%%%%%%%%%%%%%%%%%%%%%%%%%%%%%%%%

%%%%%%%%%%%%%%%%%%%%%%%%%%%%%%%%%%%%%%%%%%%%%%%%%%%%%%%%%%%%%%%%%%%%%%%%%%%%%%%%%
%               ACA EMPIEZA EL DOCUMENTO                            %
%%%%%%%%%%%%%%%%%%%%%%%%%%%%%%%%%%%%%%%%%%%%%%%%%%%%%%%%%%%%%%%%%%%%%%%%%%%%%%%%%
\begin{document}


\renewcommand{\headrulewidth}{0pt} % para que no haya linea decorativa en el header.


\author[1]{Nombre y Apellido} % defino el autor
\affil[1]{Universidad Nacional de Tres De Febrero, Buenos Aires, Argentina \newline \texttt{email@delautor.com}} % afiliacion del autor


\begin{minipage}[h]{\textwidth} % uso el entorno minipage para que el abstract este en la misma pagina que el titulo
    \maketitle
    \thispagestyle{fancy}
    \fancyhf{}
    \rhead{Fecha de entrega}
    \lhead{Materia/Congreso}
    \cfoot{\thepage}

\end{minipage}


\begin{abstract}

\textit{\lipsum[1]}

\end{abstract}

\vspace{1.5cm}% Additional space between abstract & rest of document

\begin{multicols}{2}
\section{Introducción}
En este trabajo se describen las mediciones realizadas en el marco de la materia
Electroacustica I.Las mediciones son conducidas segun dos metodos: Un método ``Clásico'' y otro metodo
moderno. Se pretende caracterizar dos parlantes en funcion de sus especificaciones.
Dichas especificaciones son:
\begin{itemize}
  \item Impedancia
  \item Sensibilidad
  \item Respuesta en frecuencia
  \item Parametros de Thiele y Small
  \item{Directividad}
\end{itemize}

Los parlantes medidos son los modelos BC 6MD38 y 18TBX100-8, de 6,5 y 18
pulgadas respectivamente.
\section{Metodologia}
En la siguiente seccion se detalla la metodologia adoptada para las distintas
mediciones llevadas a cabo en este trabajo.

\subsection{Medición de Impedancia}

\subsubsection{Método moderno}
\label{subsec: z moderno}
Para dicha medicion se empleo el paquete de software abierto Room EQ Wizard, tambien
conocido como REW. Dicho software requiere de un conexionado especifico, el cual
fue implementado mediante una solucion de Hardware diseñada a medida, resultando
en mayor agilidad a la hora de medir esta especificacion.

El arreglo instrumental utilizado es muy similar al descrito mas adelante,
en la seccion \ref{subsec: freq response}

\subsubsection{Método clásico}
\label{z clasica}
Para dicha medicion se emplea el metodo de la corriente constante detallado en \cite{Ruffa}
Dicho metodo consiste en conectar un resisteor de alto valor, entre 1 k$\Omega$ y
2 k$\Omega$, en serie con el parlante y medir la caida de tension en el mismo.
El valor relativamente alto del resistor asegura una corriente constante en el
circuito, por lo que la variacion de la tension sobre el parlante con la frecuencia
quedara reflejada directamente en el cambio de impedancia del parlante.

Medimos la resistencia electrica de la bobina $R_e$ y $R$ con
la mayor precision posible. Luego, se conecta el instrumental segun la figura
\ref{fig:setup_impedancia} :

\figura
{setup impedancia}
{Arreglo instrumental utilizado para medir impedancia de un altoparlante}
{fig:setup_impedancia}

Para permanecer en un regimen de operacion lineal, la tension de salida del
amplificador es ajustada a no mas de 2 Volts RMS.
Tambien se mide la tension sobre el resistor $V_R$ y se verifica que la misma no varie
significativamente dentro del rango de frecuencias que va desde 20 Hz hasta
500 Hz

La corriente en el circuito $i_p$ esta dada por la siguiente ecuacion:
\begin{equation}
  I_p=\frac{V_R}{R}
\end{equation}
Para medir la impedancia del altoparlante en una frecuencia dada, se ajusta la
frecuencia del generador en dicha frecuencia y se mide la tension en los bornes
del altoparlante $V_p$. El valor absoluto de la impedancia del parlante en
dicha frecuencia esta dada por:
\begin{equation}
  |Z|=\frac{V_p}{I_p}
\end{equation}

Se puede ver que dicho método resulta bastante laborioso en casos en los que se
busca obtener una gran cantidad de puntos de medición. Por lo tanto, solamente
se tomaron hasta 15 puntos por parlante siguiendo este método.


\subsection{Medicion de Sensibilidad}

Se genera ruido rosa a un nivel determinado el cual permita desarrollar 1 Watt
de potencia eficaz sobre la impedancia nominal indicada en las especificaciones
del parlante.

Una vez hallado dicho nivel, el sonometro es colocado a 1 metro de distancia del
parlante. Para medir la sensibilidad del parlante, se mide el nivel de presion
sonora en dicha posicicion.



\subsection{Medicion de Respuesta en frecuencia}
\label{sec: fresp}
Para dicha medición solamente se recurrio al metodo moderno. Se coloca un
microfono de medicion a una distancia de 1 metro del altoparlante.
Por medio de una computadora y una placa de audio se genera ruido rosa.
Para compensar la respuesta en frecuencia de dicha placa se recurrió a un lazo
de realimentacion. La señal captada por el microfono de medicion fue procesada
mediante el paquete SMAART V7.
\label{subsec: freq response}

\figura
{setup realimentacion}
{Arreglo experimental utilizado para la medicion de respuesta en frecuencia}
{feedback}


\subsection{Medicion de los parametros de Thiele y Small}
\subsubsection{Método clásico}
En primer lugar, el parlante se  suspende en espacio libre lejos de cualquier superficie
reflectante. Luego se mide con la mayor precision posible la resistencia de la
Bobina movil $R_e$ y el valor de una resistencia conocida $R_{lim}$.
Dicho reststor se conecta en serie con el parlante a medir. Este ultimo puede
considerarse una impedancia.

Se conecta el generador en serie a este circuito y de esta manera se obtiene un
divisor resistivo. Para que la impedancia del generador no resulte problematica
en la medicion, se uso una potencia de audio con baja impedancia de salida. De
esta manera, la tension en la salida de la potencia es considerada como la
tension del generador.

Se conecta el osciloscopio segun la figura \ref{fig:setup_lissajous}.
Dicha conexion permite medir con gran precision el valor de la frecuencia
de resonancia electrica del parlante, $f_s$, mediante el uso de las figuras de
Lissajous. Dichas figuras permiten medir el defasaje del canal 1 respecto al
canal 2. Cada una de las señales permiten deflectar el rayo de electrones de
manera horizontal o vertical respectivamente. Cuando la frecuencia del generador
de señales es igual a $f_s$, las partes reactiva de la impedancia del parlante
se anulan entre si, por lo que la impedancia del parlante vista por el generador
es puramente resistiva. Por lo tanto, en resonancia la tension a traves de
$R_{lim}$ estara en fase a la tension en los bornes del altoparlante, $V_{a}$.


\figura
{setup lissajous}
{Arreglo instrumental utilizado para medir $f_s$}
{fig:setup_lissajous}

Para medir $f_s$, primero se ajusta la tension de salida del generador a un valor
conocido $V_g$. Se varia la frecuencia del generador hasta visualizar una linea
recta. Dicha figura indica que la frecuencia del generador
es $f_s$. Con un miltimetro digital se mide la frecuencia de resonancia
$f_s$ y la tension en el altoparlante en resonancia $V_s$.

En función de las magnitudes medidas, se define $r_o$, como la relacion entre
la impedancia a resonancia y la resistencia de la bobina movil $R_e$:
\begin{equation}
  r_{o}=\frac{R_{lim} \cdot V_s}{Re \cdot (V_g-V_s)}
  \label{eq:ro}
\end{equation}

Luego se calcula el valor de $V_r$ dado por:

\begin{equation}
V_r = V_g \cdot \frac{\sqrt{r_0} \cdot R_e }{\left(\sqrt{r_0} \cdot R_e \right) + R_{lim}}
\end{equation}
Cuando $V_a=V_r$ se cumple que la frecuencia del generador es igual a $f_1$ o
 $f_2$  y se cumple la siguiente relacion:

\begin{equation}
  f_1 \cdot f_2 = f_s^2
  \label{eqf1f2}
\end{equation}

El cumplimiento de la ecuacion \ref{eqf1f2}
 implica que en una escala logaritmica
la curva de impedancia es simetrica.

Se fija la frecuencia del generador en $f_s$ y se le hace un pequeño ajuste
para lograr la condicion $V_a=V_r$ para valores mayores y menores a $f_s$

Con los valores obtenidos de $f_1$ y $f_2$, se verifica el cumplimiento de la
siguiente relacion:
\begin{equation}
  f_s=\sqrt{f_1\cdot f_2}
  \label{eq:verificacion}
\end{equation}
El valor de $f_s$ obtenido mediante la ecuacion \ref{eq:verificacion} no debe
diferir mas de 2 Hz del valor medido.
Conociendo $f_1$, $f_2$ y $r_0$, es posible calcular los Parametros de Thiele y
Small. Para dichos calculos se emplea una hoja de calculo.

% EXPLICAR MEDICION DE MMD
\subsubsection{MÉTODO MODERNO}
Dicha medición puede realizarse con el mismo software utilizado en

% \ref{subsec: z moderno}

para la medición de la impedancia del equipo, según el
método moderno. Por lo tanto, para dicha medición se utiliza el mismo
arreglo instrumental.

El software utilizado requiere el ingreso de ciertos parametros mas simples
de medir. Tales son: La resistencia electrica de la bobina $R_e$, el peso de
la masa agregada, la temperatura, la presión y el area efectiva del diafragma.
Esta última esta dada por:

\begin{equation}
a=\pi \cdot \left( \ \frac{d}{2} \ \right)^2
\end{equation}
Donde d es el diametro efectivo del diafragma.

Se realizaron dos mediciones según las indicaciones de este método. En la
primera se midio al altoparlante solo y en la segunda se midio el altoparlante
con una masa agregada cuyo peso es conocido.

\subsection{Medición de la directividad}
En el caso de dicha medición, solamente se estudia el parlante BC 6MD38.

\figura
{setup directividad}
{Arreglo experimental utilizado. Notese la similaridad con el arreglo presentado
en la figura \ref{feedback}}
{fig: directividad}


En este caso, el parlante se coloca sobre una superfice giratoria diseñada para
este tipo de mediciones. Luego, para minimizar el efecto de las reflexiones laterales
se coloca un microfono de referencia lo mas lejos posible de las paredes.
Se realizan 7 mediciones, con los siguientes ángulos: 0$^\circ$, 15$^\circ$,
30 $^\circ$, 45 $^\circ$, 60$^\circ$, 75$^\circ$ y 90$^\circ$. Para cada una se
emite un sweep. Se asume directividad en la simetria.

La señal obtenida por el microfono de referencia luego es procesada
en Python y MATLAB para asi obtener la directividad en tercios de octava, normalizada
respecto a 0 $^\circ$
\section{Resultados}
a
% 6
% Impedancia
% Sensibilidad
% Respuesta en frecuencia
% Directividad
% Th y S
% 18
% Impedancia
\subsection{Impedancia}
Tal como se indica indica \ref{z clasica}, la medición de impedancia segun el
método clásico resulta muy laboriosa y poco precisa en comparacion con el método digital.
Por lo tanto, resulta imposible obtener la misma
cantidad de puntos de mediciones que en el método moderno.
El pico de impedancia se del BC 18TX100 se encuentra entre los 150 y 300 Hz.
Tal es la incertidumbre, que se omiten los demas resultados de la medición de impedancia
segun el método clásico y se presentan los resultados obtenidos con el método
moderno a conticuación:

\figura
{impedancia 6MDN38-8}
{Valores obtenidos en la medicion de impedancia del BC 6MDN38-8. La línea azul
representa la curva de impedancia con masa agregada al diafragma.}
{tab: 6MDN38-8}
\subsection{Sensibilidad}
La sensibilidad medida del BC 6MDN38-8 a según el procedimiento propuesto es
de 96,2 dB SPL.
\subsection{Respuesta en frecuencia}
Tal como se indico en \ref{subsec: freq response}, la médicion de respuesta en
frecuencia del BC 6MDN38-8 fue llevada a cabo según el método digital.
Los resultados pueden verse en la figura \ref{fig: fresp}:
\figura
{rta en frecuencia 6MDN}
{Valores obtenidos en la medicion de respuesta en frecuencia del BC 6MDN38-8.
La línea azul corresponde a la medición con bafle simple}
{fig: fresp}
\subsection{Thiele y Small}
La tabla \ref{tab: tys} presenta una comparacion de los parametros de Thiele y Small
del BC 6MD38-8 medidos:

\tabla
{|c|c|c|}
{
\hline
 & Método Clásico & Método moderno \\
\hline
$f_s$ [Hz] & 164 & 164,8 \\
\hline
$R_e$ [$\Omega$] & 4,9 & 5,1   \\
\hline
$Q_{ts}$  & 0,42 & 0,512 \\
\hline
$Q_{ms}$  & 4,49 & 4,66 \\
\hline
$Q_{es}$    & 0,46 & 0,575 \\
\hline
$V_{as}$ [L] & 2,43 & 2,16 \\
\hline
$S_d$ [cm$^2$] & 133 & 132,7 \\
\hline
$\eta$ [\%{}] & 1,13 & 1,71 \\
\hline
$M_{md}$ [g] & 9,43 & 10,35 \\
\hline
$Bl$ [$T \cdot m$] & 15,7 & 9,852 \\
\hline
$C_{ms}$ [$\frac{mm}{N}$] & 0,099 & 0,086 \\
\hline
}
{Comparacion de los parametros de Thiele y Small obtenidos mediante ambos métodos}
{tab: tys}
\subsection{Directividad}
Los resultados de la medicion de directividad del BC 6MMD38-8 son presentados en
la figura \ref{fig: directividad}
\figura
{directividad}
{Directividad del BC 6MMD38-8}
{fig: directividad}
\section{Análisis de resultados}
En esta sección, se comparan los resultados obtenidos con los especificados por
el fabricante.

\subsection{Impedancia}
La figura \ref{fig: z sheet} presenta la curva de impedancia provista por el
fabricante del modelo 6MD38-8:
\figura
{z sheet}
{Curva de impedancia provista por el fabricante}
{fig: z sheet}
Se puede apreciar que la curva especificada por el fabricante es bastante similar
a la presentada en la figura \ref{fig: 6MDN38-8}, obtenida siguiendo el método
moderno.

Según las mediciones realizadas, la primera y la segunda frecuencia de resonancia
de dicho parlante es 169 Hz y 528 Hz respectivamente. Según el fabricante, la
primer frecuencia de resonancia es 130 Hz.

Tal como se indico antes, en cuanto al BC 18TX100, debido a la pobre cantidad de información recopilada
no es posible brindar una caracterización precisa de la curva de impedancia. Según los especificaciones
del fabricante, el resultado obtenido difiere por un exceso de 8,3 Hz.


Por otro lado, es evidente el
efecto de agregar una masa al diafragma apreciable en la figura \ref{fig: 6MDN38-8} :
En comparación con el diafragma sin masa agregada,
la frecuencia de resonancia es desplazada hacia la derecha y el máximo valor de
$ |Z|$ es menor.


\subsection{Sensibilidad}
En cuanto a la medición de sensibilidad, el resultado obtenido difiere por 0.2 dB
al valor de sensibilidad especificado.

\subsection{Respuesta en frecuencia}
La figura \ref{fig: fresp sheet 6m} muestra la respuesta en frecuencia especificada del BC 6MMD38-8

\figura
{fresp sheet 6m}
{Respuesta en frecuencia especificada para el BC 6MMD38-8}
{fig: fresp sheet 6m}

En el caso del altoparlante BC 6MMD38-8, al comparar la figura
\ref{fig: fresp sheet 6m} con la figura \ref{fig: fresp}
se puede notar que la curva de respuesta en frecuencia obtenida mediante el método moderno
es bastante similar a la especificada por el fabricante.

Ademas, se observa que la banda de operación medida es muy similar a la especificada
por el fabricante.

La figura \ref{fig: 6MDN38-8} también demuestra mejoras en la respuesta en frecuencia
al agregar un bafle simlpe al altoparlante.


\subsection{Parámetros de Thiele y Small}
La tabla \ref{tab: comparacion tys} es una extension de la tabla \ref{tab: tys}.
Esta última incorpora los valores especificados de los Parámetros de Thiele y small
para el BC 6MDN38-8.

\tabla
{|l|c|c|c|}
{
\hline
 & Clásico & Moderno & Fabricante\\
\hline
$f_s$ [Hz] & 164 & 164,8 & 130 \\
\hline
$R_e$ [$\Omega$] & 4,9 & 5,1 & 5,7  \\
\hline
$Q_{ts}$  & 0,42 & 0,512 & 0,44 \\
\hline
$Q_{ms}$  & 4,49 & 4,66 & 3,7 \\
\hline
$Q_{es}$    & 0,46 & 0,575 & 0,49 \\
\hline
$V_{as}$ [L] & 2,43 & 2,16 & 3 \\
\hline
$S_d$ [cm$^2$] & 133 & 132,7 & 132 \\
\hline
$\eta$ [\%{}] & 1,13 & 1,71 & 1,4 \\
\hline
$M_{md}$ [g] & 9,43 & 10,35 & 12\\
\hline
$Bl$ [$T \cdot m$] & 15,7 & 9,852 & 10,5\\
\hline
$C_{ms}$ [$\frac{mm}{N}$] & 0,099 & 0,086 & NA \\
\hline
}
{Comparación de los valores obtenidos y los especificados}
{tab: comparacion tys}

En 7 de los 11 Parámetros estudiados, el resultado mas próximo al valor especificado
fue obtenido mediante el método clásico.

\subsection{Directividad}
Lamentablemente, el fabricante no especifica la directividad de sus equipos
por lo que la medición desarrollada no puede ser comparada.




\section{Conclusión}
Los dos métodos utilizados poseen ciertas ventajas y desventajas. Si bien hoy en día
las mediciones computarizadas suelen ser preferidas ante las mediciones manuales,
en este caso se trabajó con ambas y se observaron las ventajas y desventajas de
ambos métodos.

A la hora de caracterizar la curva de impedancia de un parlante, el método
clásico probo ser mas laborioso y demandante que el método digital. De todos modos
este método resulta bastante preciso si lo que se busca es el comportamiento del
altoparlante para una serie de frecuencias discretas.

En cambio, las técnicas digitales han demostrado ser mas rápidas de implementar
pero no siempre dan resultados acertados, suponiendo que el fabricante ha especificado
sus productos debidamente. Esto se puede deber a varios factores, pero el mas
relevante probablemente provenga de el software utilizado.

Por esta misma razón resulto mas precisa la medición de la frecuencia de resonancia
de un parlante mediante el método clásico

En síntesis, si bien el método clásico utilizado para medir Parámetros de Thiele y small
resulto mas laborioso que el método computarizado, es el primero el que entrego
mejores resultados. No se puede decir lo mismo sobre las mediciones de impedancia.
En este caso, claramente el método digital resulto ser una gran herramienta.
\printbibliography

\end{multicols}





\end{document}
